\begin{abstract}

Computational Workflows are widely used in data analysis, enabling innovation and decision-making for the modern society. In many domains the analysis components are numerous and written in multiple different computer languages by third parties. These polylingual workflows are common in many industries and dominant in fields such as bioinformatics, image analysis, and radio astronomy.
%
However, in practice many competing workflow systems exist, severely limiting portability of such workflows, thereby hindering the transfer of such workflows between different systems, between different projects and different settings, leading to vendor lock-ins and limiting their generic re-usability.   
%
Here we present the Common Workflow Language (CWL) project which produces free and open standards for describing command-line tool based workflows. The CWL standards provide a common but reduced set of abstractions that are both used in practice and implemented in many popular workflow systems. The CWL language is declarative, which allows expressing computational workflows constructed from diverse software tools, executed each through their command-line interface. Being explicit about the runtime environment and any use of software containers enables portability and reuse. The CWL project is not specific to a particular analysis domain, it is community-driven, and it produces consensus-built standards.  
%
Workflows written according to the CWL standards are a reusable description of that analysis that are runnable on a diverse set of computing environments. These descriptions contain enough information for advanced optimization without additional input from worklfow authors.
%
The CWL standards support polylingual workflows, enabling portability and reuse of such workflows, easing for example scholarly publication, fulfilling regulatory requirements, collaboration in/between academic research and industry, while reducing implementation costs. CWL has been taken up by a wide variety of domains, and industries and support has been implemented in many major workflow systems.

%% [Check if more conclusions need to be added from (1-5) in the conclusion section]


  %% max 400 words; ~279 now
\end{abstract}

